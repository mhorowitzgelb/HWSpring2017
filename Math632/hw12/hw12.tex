\documentclass{article}
\usepackage{amsmath}
\usepackage{amsfonts}

\title{HW 12}
\author{Max Horowitz-Gelb}
\date{4/17/17}

\begin{document}
\maketitle
\section*{Q1}
\subsection*{(1)}
Since $\lambda_i = \sum_{i \neq j} q(i,j)$ then
$$
\lambda_i = q(i,i+1) + q(i, i-1) = \lambda + \mathbf{1}_{\{0 \leq i \leq s\}}i\mu + \mathbf{1}_{\{i \geq s\}}s\mu 
$$

\subsection*{(2)}
Then,
$$
Q(i,j) = 
\begin{cases}
\lambda & j = i +1 \\
i \mu  & 0 \leq i \leq s, \qquad j = i -1 \\
s \mu & i \geq s, \qquad j = i -1 \\
-\lambda_i & i = j 
\end{cases}
$$

\section*{Q2}
Let $\lambda$ be the rate of $N(t)$.
First note that clearly 
$q(i,j) = 0$ for $i -j \geq 2$ or $j < i$
then for $j = i + 1$
$$
q(i,j) = \lim_{t \to 0} \frac{e^{-\lambda t}(\lambda t)^{j-i}}{(j-i)! t} = \lim_{t \to 0} \frac{e^{-\lambda t}\lambda t}{t} = \lambda
$$

Then we have
$$
Q(i,j) = \begin{cases}
\lambda & j = i +1 \\
-\lambda & j = i \\
0 & else
\end{cases}
$$


\end{document}