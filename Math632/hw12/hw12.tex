\documentclass{article}
\usepackage{amsmath}
\usepackage{amsfonts}

\title{HW 12}
\author{Max Horowitz-Gelb}
\date{4/17/17}

\begin{document}
\maketitle
\section*{Q1}
\subsection*{(1)}
Since $\lambda_i = \sum_{i \neq j} q(i,j)$ then
$$
\lambda_i = q(i,i+1) + q(i, i-1) = \lambda + \mathbf{1}_{\{0 \leq i \leq s\}}i\mu + \mathbf{1}_{\{i \geq s\}}s\mu 
$$

\subsection*{(2)}
Then,
$$
Q(i,j) = 
\begin{cases}
\lambda & j = i +1 \\
i \mu  & 0 \leq i \leq s, \qquad j = i -1 \\
s \mu & i \geq s, \qquad j = i -1 \\
-\lambda_i & i = j 
\end{cases}
$$

\section*{Q2}
Let $\lambda$ be the rate of $N(t)$.
First note that clearly 
$q(i,j) = 0$ for $i -j \geq 2$ or $j < i$
then for $j = i + 1$
$$
q(i,j) = \lim_{t \to 0} \frac{e^{-\lambda t}(\lambda t)^{j-i}}{(j-i)! t} = \lim_{t \to 0} \frac{e^{-\lambda t}\lambda t}{t} = \lambda
$$

Then we have
$$
Q(i,j) = \begin{cases}
\lambda & j = i +1 \\
-\lambda & j = i \\
0 & else
\end{cases}
$$

Then by the backward Kolmogorov equation 
$$
p'_t = Qp_t
$$
This is a partial differential equation whose solution is 
$$
p_t = e^{tQ}
$$

\section*{Q3}

First note that for $i \neq j$
$$
q(i,j) = \lim_{t \to 0} \frac{p_t(i,j) u(i,j)}{t}
$$
which by what we have shown in two is equal to simply $\lambda u(i,j)$
Then $Q$ is 
$$
Q(i,j) = \begin{cases}
\lambda u(i,j) & i \neq j \\
-\sum_{k \neq j} \lambda u(j,k) & i = j 
\end{cases} 
$$

Then 
$$
p'_t(i,j) =  \lim_{h \to 0} \frac{p_{t+h}(i,j) - p_t(i,j)}{h}
$$
$$
= \lim_{h \to 0} \frac{\sum_{k} \big(p_t(i,k) p_h(k,h)\big) - p_t(i,j)}{h}
$$
$$
= \lim_{h \to 0} \frac{\sum_{k\neq j} \big(p_t(i,k) p_h(k,h)\big) + p_t(i,j)(p_h(j,j)-1)}{h}
$$
$$
= \lim_{h \to 0} \frac{\sum_{k\neq j} \big(p_t(i,k) p_h(k,h)\big)}{h} + \lim_{h \to 0} \frac{p_t(i,j)(p_h(j,j)-1)}{h}
$$
$$
= \sum_{k\neq j} p_t(i,k) q(k,h) - \lim_{h \to 0} \frac{p_t(i,j)(\sum_{k\neq j} p_h(j,k)}{h}
$$
$$
= \sum_{k\neq j} p_t(i,k) q(k,h)  + p_t(i,j)(-\sum_{k\neq j} q(j,k))
$$
$$
= \sum_{k\neq j} p_t(i,k) \lambda u(k,j)  + p_t(i,j)\big(-\sum_{k\neq j} \lambda u(j,k)\big)
$$
$$
= p_t(i,\cdot) Q(\cdot,j)
$$

\section*{Q4}
Let $L$ be the event that an inpatient person leaves, $B$ the event that a bank teller finishes with someone and $A$ the event that a new person enters the bank. 
First note that even with this new addition to the problem, still $q(i,j) = 0$ for $j - i \geq 2$. 

Then 
$$
q(n, n+1) = \lim_{t \to 0} \frac{P(A)}{t} = \frac{e^{-\lambda t}\lambda t}{t}  =\lambda
$$
and
$$
q(n, n-1) = \lim_{t \to 0} \frac{P(L, \bar{B}) + P(B, \bar{L})}{t} 
$$
as $t$ goes to 0 the $P(\bar{B}) \to P(\bar{L}) \to 0$ so the above becomes 
$$
\lim_{t \to 0} \frac{P(L) + P(B)}{t} 
$$

$$
= \lim_{t \to 0} \frac{ \bigg( e^{-\max(n-s,0)\xi t} \max(n-s,0)\xi t \bigg ) + 
\bigg( e^{-\min(n,s)\mu t} \min(n,s)\mu t \bigg)}{t}
$$
$$
= \max(n-s,0)\xi + \min(n,s)\mu
$$
We might think that $q(n, n+2) > 0$, but this is not true
since it would require $L$ and $B$ to happen at the exact same time,
$$
lim_{t \to 0} \frac{P_t(L, B)}{t} = lim_{t \to 0} \frac{P_t(L)P_t(B)}{t} 
$$
$$
= lim_{t \to 0}\frac{e^{-\xi\mu \min(n,s) \max(n-s,0) t^2}t^2 \xi \mu \min(n,s) \max(n-s,0)}{t} = 0
$$

Therefore,
$$
Q(i,j) = \begin{cases}
\lambda & j = i +1 \\
\max(i-s,0)\xi + \min(i,s)\mu & j = i -1\\
0 & |i-j| \geq 2 \\
-\lambda - \max(i-s,0)\xi - \min(i,s)\mu & i = j
\end{cases}
$$

\end{document}