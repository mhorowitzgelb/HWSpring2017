\documentclass{article}
\usepackage{amsmath}
\usepackage{parskip}

\title{HW Week 3}
\author{Max Horowitz-Gelb}
\date{February 2nd, 2017}
\setlength{\parindent}{0pt}
\begin{document}
\maketitle
\section*{Q1}
We have shown in class that this probability can be rewritten as for arbitrary $n \geq 1$ as, 
\[
P(X_{n+2} = 3, X_{n+4} = 4 | X_{n+1} = 9 , X_n = 8)
\]
Then since $X$ is a THMC this can be written simply as,
\[
p(9,3)*p(3,4)^2
\]

\section*{Q2}
\subsection*{a}
If $V = \text{max} \{T,U\}$ then $V$ is a stopping time. This is because there are only two cases.

\textbf{Case 1}: $T \geq U$

Since $T$ and $U$ are stopping times, we can check this with only $X_1 ... X_U$, which implies then that we can tell this with only $X_1 ... X_T$. Then in this case $V = T$ and since $T$ is a stopping time, $V$ can be determined with only $X_1 ... X_{V}$.

\textbf{Case 2}: $T < U$

Since $T$ and $U$ are stopping times, we can check this with only $X_1 .. X_T$ which implies we can tell this with $X1 ... X_U$. Then in this case $V = U$ and since $U$ is a stopping time, $V$ can be determined with only $X_1 ... X_{V}$.

Then by definition $V$ is a stopping time. 
\subsection*{b}
If $V = \text{min}\{T, U\}$ then again there are the two same cases.

\textbf{Case 1}: $T \geq U$

Again we can tell this with only $X_1 ... X_U$. Then $V = U$ and since $U$ is a stopping time, we can know $V$ with only $X_1 ... X_V$.

\textbf{Case 2}: $T < U$.

Again we can determine if this inequality is true with only $X_1 ... X_T$. And $V = T$, and since $T$ is a stopping time, then we can determine $V$ with only $X_1 ... X_V$.

Therefore $V$ is a stopping time.
\end{document}