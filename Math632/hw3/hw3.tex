\documentclass{article}
\usepackage{amsmath}
\usepackage{amsfonts}
\usepackage{parskip}

\title{HW Week 3}
\author{Max Horowitz-Gelb}
\date{February 2nd, 2017}
\setlength{\parindent}{0pt}
\begin{document}
\maketitle
\section*{Q1}
We have shown in class that this probability can be rewritten as for arbitrary $n \geq 1$ as, 
\[
P(X_{n+2} = 3, X_{n+4} = 4 | X_{n+1} = 9 , X_n = 8)
\]
Then since $X$ is a THMC this can be written simply as,
\[
p(9,3)*p^2(3,4)
\]

\section*{Q2}
\subsection*{a}
If $V = \text{max} \{T,U\}$ then $V$ is a stopping time. This is because there are only two cases.

\textbf{Case 1}: $T \geq U$

Since $T$ and $U$ are stopping times, we can check for this case with only $X_1 ... X_U$, which implies that we can check this case with only $X_1 ... X_T$. Then if we are in this case, $V = T$ and since $T$ is a stopping time, $V$ can be determined with only $X_1 ... X_{V}$.

\textbf{Case 2}: $T < U$

Since $T$ and $U$ are stopping times, we can check this case with only $X_1 .. X_T$ which implies we can tell this case with $X1 ... X_U$. Then in this case, $V = U$ and since $U$ is a stopping time, $V$ can be determined with only $X_1 ... X_{V}$.

Then by definition $V$ is a stopping time. 

\subsection*{b}
If $V = \text{min}\{T, U\}$ then again there are the two same cases.

\textbf{Case 1}: $T \geq U$

Again we can tell if we are in this case with only $X_1 ... X_U$. Then if we are in this case, $V = U$ and since $U$ is a stopping time, we can know $V$ with only $X_1 ... X_V$.

\textbf{Case 2}: $T < U$.

Again we can determine if we are in this case with only $X_1 ... X_T$. And if we are in this case, $V = T$, and since $T$ is a stopping time, then we can determine $V$ with only $X_1 ... X_V$.

Therefore $V$ is a stopping time.

<<<<<<< HEAD
=======
\section*{Q3}
Let $n$ be given. First note that by definition of a THMC
\[
P(X_{T+2} = j | X_T = i , T = n) = P(X_{n+2} = j | X_n = i, past) = p^2(i,j)
\]
implies,
\[
\forall n'
P(X_{T+2} = j | X_T = i , T = n') = p^2(i,j)
\]
Then note that $P(X_{T+2} = j | X_T = i)$ can be rewritten as,
\[
\sum_{n'} P(X_{T+2} = j | X_T = i, T=n')*P(T=n')
\]
Which by what we've shown above can be rewritten as,
\[
= p^2(i,j) * \sum_{n'} P(T=n')
\] 
\[
=p^2(i,j)
\]

\section*{Q4}
Let $X_0 = E$, and $T = min\{n \geq 1 : X_n = E\}$
Then 
\[
P(T = \infty) = p(E,N) * \lim_{n \to \infty} \prod_{i = 2}^{n} p(N,N)
\]
\[
= 0.9 * \lim_{n \to \infty} \prod_{i = 2}^{n} 0.2
\]
\[
= 0
\]
Therefore $P(T < \infty) = 1$ and $E$ is a recurrent state.

\section*{Q5}
Let $E : \{X_1 .. X_n\} \mapsto \mathbb{R}$ and $N : \{X_1 ... X_n\} \mapsto \mathbb{R}$ be functions such that $E(\{X_1, ... X_n\})$ equals the number of $E$ states in the set $\{X_1 ... X_n\}$ and $N(\{X_1, ... X_n\})$ equal the number of $N$ states in $\{X_1, ... X_n\}$. 

Then $T = n$ if and only if $E(\{X_1,... X_n\}) \geq 1$, $N(\{X_1,... X_n\}) \geq 2$ and $E(\{X_1,... X_{n-1}\}) = 0$ or $N(\{X_1,... X_{n-1}\}) < 2$. Since this is a function of only $X_1 ... X_n$ then $T$ is a stopping time. 

Now we consider $P(T = +\infty | X_0 = E)$.
This is bounded by,
\[
P_{n \to \infty}(N(X_1 ... X_n) < 2) + P_{n \to \infty}(E(X_1 ... X_n) < 1)
\]
For the left part of this bound,
\[
P_{n \to \infty}(N(X_1 ... X_n) < 2) = P_{n \to \infty}(N(X_1 ... X_n) = 0) + 
P_{n \to \infty}(N(X_1 ... X_n) = 1)
\]
\[
P_{n \to \infty}(N(X_1 ... X_n) = 0) = \lim_{n \to \infty} p(E,E)^n =\lim_{n \to \infty} 0.1^n = 0
\]
\[
P_{n \to \infty}(N(X_1 ... X_n) = 1) = \lim_{n \to \infty} {{n}\choose{1}} * p(E,N) * p(N,E) * p(E,E)^n
\]
\[
= \lim_{n \to \infty} n * 0.9 * 0.2 * 0.1 ^n = 0
\]
And for the right part of the bound,
\[ 
P_{n \to \infty}(E(X_1 ... X_n) < 1) = P_{n \to \infty}(E(X_1 ... X_n) = 0)
\]
\[
= \lim_{n \to \infty} p(E,N) * p(N,N)^n = \lim_{n \to \infty} 0.9 * 0.8 ^ n = 0
\]

Therefore $P( T = +\infty | X0 = E) = 0$




>>>>>>> 9415f79994a2e076bbb6afce065c69c392466210


\end{document}