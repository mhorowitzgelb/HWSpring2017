 

\documentclass{article}
\usepackage{amsmath}
\title{HW1}
\author{Max Horowitz-Gelb horowitzgelb@wisc.edu}
\date{1/19/17}

\begin{document}
\maketitle
\section*{Q1}
Let $X_1 ... X_{40}$ be a set of random variables such that $X_i = 1$ is the event that box $i$ is empty
after all $80$ balls have been placed, and $X_i = 0$ otherwise.
Then the number of boxes empty after all balls are placed, which let's call $T$, is 
\[
T = \sum_{i=1}^{40} X_i
\]
Then,
\[
E[T] = E[\sum_{i=1}^{40} X_i] = \sum_{i=1}^{40} E[X_i]
\]
The expected value of any $X_i$ is the same and can be calculated as,
\[
E[X_i] = 1*P(X_i = 1) + 0 * P(X_i = 0) = P(X_i = 1) = \frac{39}{40}^{80} = 0.13193780538
\]
Therefore,
\[
E[T] = \sum_{i=1}^{40} E[X_i] = 40 * 0.13193780538 = 5.27751221548
\]
A similar procedure is used to find the vaiance of $T$.
We know that the variance of $T$ is equal to $E[T^2] - E[T]^2 = E[T^2] - 27.85213518454$.
Then,
\[
E[T^2] = E[ \sum_{i=1}^{40}\sum_{j=1}^{40}X_i X_j] = \sum_{i=1}^{40}\sum_{j=1}^{40}E[X_i X_j]
\]
\[
E[X_iX_j] = 1 * P(X_iX_j = 1) + 0 * P(X_iX_j = 0) = P(X_iX_j = 1)
\]
If $i = j$ then $X_iX_j = 1$ if and only if $X_i = 1$ and then,
\[
P(X_iX_j = 1) = P(X_i = 1) = 0.13193780538
\]
If $i \neq j$ then
\[
P(X_iX_j = 1) = \frac{38}{40}^{80} = 0.01651537438
\]
Therefore, 
\[
E[T^2] = 40 * P(X_iX_j = 1 | i = j) + 40*39 * P(X_iX_j=1|i\neq j) 
\]
\[
= 40 * 0.13193780538 + 40*39*0.01651537438 = 31.041496248
\]
So,
\[
Var(T) = E[T^2] - E[T]^2 = 31.041496248 - 27.85213518454 = 3.18936106346
\]
\section*{Q2}
If $X$ is a uniform random variable over $[0,\frac{\pi}{2}]$ then $X$ has a pdf,
\[
f(x) = \frac{2}{\pi} \text{ for } x \in [0,\frac{\pi}{2}]
\]
Therefore,
\[
E[sin(X)] = \int_{0}^{\frac{\pi}{2}} sin(x)f(x)dx = \frac{2}{\pi}
\]
\section*{Q3}
At time $n$ there are $X_n$ white balls in the left urn, $5-X_n$ black balls in the left urn, $5-X-n$ white balls in the right urn and $X_n$ black balls in the right urn. To calculate the transition probability $P(X_{n+1} = j | X_n = i)$ we must consider 3 cases. If $i = j$, then either we must swap a black ball from the left urn with a black ball from the right urn or a white ball from the left urn and a white ball from the right urn. The probability on the black balls is ,
\[
\frac{i}{5} * \frac{5-i}{5} = \frac{5i - i^2}{25}
\]
and the probability from the white balls is the same,
\[
\frac{5-i}{5} *\frac{i}{5}= \frac{5i - i^2}{25}
\]
Therefore the total probability for $j = i$ is, 
\[
2*\frac{5i - i^2}{25}
\]
If $j = i-1$ we must select a white ball from the left urn and a black ball from the right urn. The probability for this is 
\[
\frac{i}{5} * \frac{i}{5} = \frac{i^2}{25}
\]
Finally if $j = i + 1$ then we must select a black ball from the left urn and a white ball from the right urn. This has a probability,
\[
\frac{5-i}{5} * \frac{5-i}{5} = \frac{(5-i)^2}{25}
\]

So our complete set of transition probabilities is,
\[
P(X_{n+1} = j | X_n = i) = \begin{cases}
2*\frac{5i - i^2}{25} , i = j\\
\frac{i^2}{25} , j = i -1\\
\frac{(5-i)^2}{25} , j = i +1\\
0 , \text{ else}
\end{cases}
\]

\end{document}
