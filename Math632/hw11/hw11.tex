\documentclass{article}

\title{HW 11}
\author{Max Horowitz-Gelb}
\date{4/9/17}

\begin{document}
\maketitle
\section*{Q1}
Let $N(t)$ be the number of customer arrivals after time $t$. 
Then since $F$ is independent of $G$ then by the expectation of a random sum,
$$
E[W(t)] = E[N(t)]\mu_G
$$
then our limit becomes
$$
\lim_{t \to \infty} \frac{E[N(t)]\mu_G}{t} 
$$
$$
= \lim_{t \to \infty} \frac{E[N(t)]\mu_G}{N(t)}\frac{N(t)}{t}
$$
which by theorem 3.1 is
$$
\frac{\mu_G}{\mu_F} \lim_{t \to \infty} \frac{E[N(t)]}{N(t)} 
$$
which by Law of Large Numbers is simply
$$
\frac{\mu_G}{\mu_F}
$$

\section*{Q2}
\subsection*{a}
Each child's shooting time follows a geometric distribution with expected value $1/p_i$
Then let $t_i$ be the total time for all the kids to shoot their $i^{th}$ time. 
Then for arbitrary kid $j$ for time $i$ let 
$s_i$ be the time it takes kid $j$ to get a basket and $u_i$  the total time it takes the other two kids to make their baskets. 
Then since $(s_i, u_i)$ are independent, then by theorem 3.4 the limiting fraction of the time spent by player $j$ shooting is 
$$
E[s_i] / (E[u_i] + E[s_i])
$$  

\subsection*{b}
so for $p_1 = 2/3, p_2 = 3/4, p_3 = 4/5$ 

we have $t_1 = X_1 + X_2 + X_3$ where $X_1 \sim Geometric(2/3), X_2 \sim Geometric(3/4), X_3 \sim Geometric(4/5)$ 

then expected fraction of the time player 1 is shooting is,
$$
E[X_1]/(E[X_1] + E[X_2 + X_3]) = \frac{3/2}{3/2 + 4/3 + 5/4} = 0.36735
$$

the expected fraction for player 2 is 
$$
0.32653
$$

and for player 3 is,
$$
0.30612
$$

\section*{Q3}
\subsection*{a}

Let $t_i = d_i + w_i$ where $d_i$ is the time it takes to deal out the $i_{th}$ baggy of coke, and $d_i \sim G$ is the time the dealer must wait until another customer arrives. Since a Poisson process is memory-less, $w_i \sim exp(\lambda)$

Then since the $t_i$ are independent
$$E[t_i] = E[t_1] = E[d_1] + 1/\lambda$$ 

and the rate at which the dealer makes sales is
$$
\frac{1}{E[d_1] + 1/\lambda}
$$

\subsection*{b}
Let $N(t), N_s(t), N_n(t)$ be the customers who arrived, who received a sale and who did not receive a sale after time $t$ respectively. 

Then the limiting fraction of customers who did not receive a sale is 
$$
\lim_{t \to \infty} \frac{N_n(t)}{N(t)}
=
\lim_{t \to \infty} \frac{N(t) - N_s(t)}{N(t)} = 1 - \lim_{t \to \infty} \frac{N_s(t)}{N(t)}
$$
$$
= 1 - \lim_{t \to \infty} \frac{N_s(t)}{t}\frac{t}{N(t)}
$$
which by theorem 3.1 and what we've shown previously in part a becomes,
$$
1 - \frac{1}{E[d_1] + 1/\lambda} \frac{1}{\lambda}
$$

\section*{Q4}
\subsection*{a}
Let $t_i \sim exp(0.5)$ the time of interval $i$ and let $t_i = r_i + s_i$ where $r_i$ is the amount of sleep had in interval $i$ and is equal to $max(t_i - 0.6, 0)$ 
Since $(r_i, s_i)$ are independent then by theorem 3.4 the limiting fraction of time sleeping is
$$
\frac{E[r_1]}{E[r_1] + E[u_1]} = \frac{E[r_1]}{E[t_1]} = \frac{P(t_1>0.6)E[t_1 - 0.6| t_1 > 0.6]}{E[t_1]} = P(t_1 > 0.6) = 0.74082
$$
\subsection*{b}
Because of independence of the $(r_i, u_i)$  the expected time awake is
$$
E[u_i] = E[u_1] = E[t_1] - E[r_1] = 1.25918
$$


\end{document}