\documentclass{article}
\title{HW 9}
\date{3/27/17}
\author{Max Horowitz-Gelb}
\usepackage{amsmath}
\usepackage{amsfonts}
\usepackage{parskip}


\begin{document}

\setlength{\parskip}{5pt}
\setlength{\parindent}{0pt}

\maketitle
\section*{Q1}
\subsection*{a}
Let $T_0 = 0$ represent the time at 8 AM. Then the probability that no people show up from 8 to 10 AM is equal to 
$$
P(T_1-T_0 > 2) = 1 - F_{exp(3)}(2) = 0.00248
$$ 

\subsection*{b}
$T_n$ is a Poisson process with rate 3 so by definition, $T_1 - T_0 \sim exp(3)$

\section*{Q2}
\subsection*{a}
Let $T_n$ be a Poisson process of cars passing through. Then let $T_0 = 0$ be the last time a car passed before the dear crosses the road. Then let $\alpha \geq 0$ be the time at which the dear crosses the road.
Then the probability that a car hits the dear is equivalent to 
$$
P(T_{k} - \alpha < 1/12)
$$ where $T_k = min\{T_n : T_n > \alpha\}$
Because of the memoryless of the exponential distribution this is equal to, 
$$
\sum_{i > 0} p(k = i)F_{exp(6)}(1/12) = F_{exp(6)}(1/12) \sum_{i > 0} p(k = i) = F_{exp(6)}(1/12) = 0.39347
$$
\subsection*{b}
If the deer only needs 2 seconds to cross the road then the problem is the same and the probability of getting hit is,
$$
F_{exp(6)}(1/30) = 0.18127
$$

\section*{Q3}
If $A$ is the total number of muons in the day, then $A$ is a sum of 3 Poisson random variables $X_1, X_2, X_3$,representing different the different signal rates throughout the day.We then have that 

$X_1 \sim 8 Poisson(240), \qquad X_2 \sim 9Poisson(360), \qquad X_3 \sim 7Poisson(420)$

Then since a sum of Poissons is Poisson,
$$
A \sim Poisson(8100)
$$

And therefore,
$$
Var(A) = 8100
$$


\section*{Q4}
Let $Y_1, ... Y_N$ be a set of iid random variables representing the amount of money withdrawn from each customer, each with mean 30 and variance 400. Then let $N$ be a Poisson random variable with rate 80 representing the number of customers over the span of 8 hours. 
Finally let $S = \sum_{i = 1} ^ N Y_i$ be the total amount of money withdrawn
Then since $|E[Y_i]| = 30 < \infty$ and $E[N] = 80 < \infty$ 
$$
E[S] = E[N] \cdot E[Y_i] = 80 * 30 = 2400
$$
and since $N$ is Poisson with finite rate then,
$$
Var(S) = 80 * E[Y_i^2] = 80 * ( Var(Y_i) + E[Y_i]^2) = 80*(400 + 900) = 104000
$$
$$
Stdev(S) = 104000^{0.5} = 322.49031
$$


\section*{Q5}

Let $X_1, ... X_N \in \{E, \bar E \}$ be a set of independent random variables representing whether each person is enthusiastic or not where $N$ is a Poisson random variable with mean 60.


Then let $N_E = \lvert \{ X_i : X_i = E \} \rvert$ and
$N_{\bar E} = \lvert \{ X_i : X_i = \bar E \}\rvert$ . Then since $X_1, ... X_N$ are iid, then $N_E \sim Poisson(40)$ and 
$N_{\bar E} \sim Poisson(20)$

Then let $Y_{E,1}, ... Y_{E,N_E}$ be the random variables of work done by enthusiastic workers and let $S_E = Y_{E,1}+ ...+ Y_{E,N_E}$.

Then since $Y_{E,i}, N_E$ have finite mean and variance. 
$$
E[S_E] = E[N_E] \cdot E[Y_{E,i}] = 40 * 10 = 400
$$
$$
Var[S_E] = E[N_E] * E[Y_{W,i}^2] = 40 * (25 + 100) = 5625
$$

Then let $Y_{\bar E,1}, ..., Y_{\bar{E},N_{\bar E}}$ be the random variables of work done by lazy workers and let 
$S_{\bar E} = Y_{\bar E,1}+ ...+ Y_{\bar E,N_{\bar E}}$.

Then since $Y_{\bar E, i}, N_{\bar E}$ have finite mean and variance.

$$
E[S_{\bar E}] = E[N_{\bar E}] \cdot E[Y_{\bar E, i}] = 20 * 3 = 60
$$
$$
Var[S_{\bar E}] = E[N_{\bar E}] * E[Y_{\bar E, i}^2] = 20 (4 + 9) = 260
$$

Then let $S = S_E + S_{\bar E}$ be the total number of cans picked up.
Then since $S_E \bot S_{\bar E}$

$$
E[S] = E[S_E] + E[S_{\bar E}] = 400 + 60 = 460
$$

$$
Var(S) = Var(S_E) + Var(S_{\bar E}) = 5625 + 260 = 5885
$$
$$
Stdev(S) = 76.71375
$$

\section*{Q6}
Let $N_T(t), N_S(t) $ be the number of trout and salmon caught after time $t$ respectively. Since the probability of each catch being a trout or a salmon is independent of other catches, then $N_T(t)$ and $N_S(t)$ are two independent Poisson process, with 

$N_T(t) \sim Poisson(0.6 * 2 * t), \qquad N_S(t) \sim Poisson(0.4 * 2 * t)$

So, 
$$
P(N_T(2.5) = 2, N_S(2.5)) = f_{Poisson(3)}(2)*f_{Poisson(2)}(1) = 0.06064
$$

\section*{Q7}
\subsection*{a}
Since whether she writes a speeding or a DWI ticket is independent of all other tickets. We can represent the tickets she writes by two independent Poisson processes. 
Let $N_{S}(t), N_{D}(t)$ be two independent Poisson processes of the number of speeding and DWI tickets written after time $t$ respectively. 
Then
$$
N_S(t) \sim Poisson((2/3) * 6 * t) \qquad N_D(t) \sim Poisson((1/3) * 6 * t)
$$  

Then let $S_S(t) = 100 * N_S(t)$ be the total revenue made from speeding tickets after time $t$. 

Then, 
$$E[S_S(1)] = 100 * E[N_S(1)] = 100 * 4 = 400$$
and,
$$Var(S_S(1)) = 100^2 * Var(N_S(1)) = 100^2 4 = 40000
$$

Then let $S_D(t) = 400 * N_D(t)$ be the the the total revenue made from DWI tickets after time $t$,

Then,
$$
E[S_D(1)] = 400 * E[N_D(t)] = 400 *2 = 800
$$
and,
$$
Var(S_D(1)] = 400^2 Var(N_D(t)) = 400^2 * 2 = 320000
$$

Then let $S(t) = S_D(t) + S_S(t)$ be the total revenue made after time $t$.

Then since $S_D(t) \bot S_S(t)$,

$$
E[S(1)] = E[S_D(1)] + E[S_S(1)] = 400 + 800 = 1200
$$
and,
$$
Var(S(1)) = Var(S_D(1)) + Var(S_S(1)) =  40000 + 320000 = 360000
$$
$$
Stdev(S(1)) = 600
$$

\subsection*{b}




\end{document}