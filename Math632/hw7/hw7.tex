\documentclass{article}
\title{HW 7}
\author{Max Horowitz-Gelb}
\date{3/3/17}

\usepackage{amsmath}
\usepackage{amsfonts}

\begin{document}
\maketitle
\section*{Q1}
\section*{Q2}
First without loss of generality we may assume the starting position is at time 1 since otherwise we could simply rename all the states by rotating the clock until the starting position was 1. Also note that all states communicate with state 1 since there is at most a shortest path of length 6 from any state to 1 with a probability of $0.5^{6}$ of occurring.
Then using this, we may apply theorem 1.28 and say that
if $g(x)$ is the expected number of transitions required to first get to state 1, then
$g(1) = 0$ and for $x \neq 1$
\[
g(x) = 1 + \sum_y p(x,y)g(y) = 1 + \sum_y r(x,y)g(y)
\]
where $r$ is $p$ restricted to $\{2 .. 12\}$.
Then we can solve
\[
g[2 .. 12] = (I - r)^{-1} \vec{1}
\]
$g$ turns out to be ,
\[
[0, 11,  20,  27,  32,  35,  36,  35,  32,  27,  20,  11]^T
\]

Then since with probability 1 the starting state will transition to 2 or 12, then the expected the expected number of steps it will take $X_n$ to return to the starting position is
\[
1 + 0.5 * g(2) + 0.5 * g(12) = 1 + 0.5 * 11 + 0.5 * 11 = 12
\]

\section*{Q3}
\subsection*{a}
If we represent the the kings moves as a random walk through a graph then from example 1.33, we know that for some positive constant $c$, $\pi$ will have the detailed balance condition if 
\[
\pi(u) = c*d(u)
\]
Then we must solve for $c$ so that $\pi$ sums to 1. There are three different board positions in a chess board. 
4 corner positions with a degree of 3. 24 side positions with degrees of 5. And 36 interior positions with degree of 8
So $c$ should be $(4*3 + 24*5 + 36*8)^{-1} = 1/420$.
So,
\[
\pi(u) = \begin{cases} 
3/420 & u \text{ is a corner}\\
5/420 & u \text{ is a side}\\
8/420 & u \text{ is  an interior}
\end{cases}
\]

\subsection*{b}
Then note that for any board position the king can return to that board position using 2 or 3 moves. These have a gcd of 1 therefore $X_n$ is aperiodic. Also clearly all the board positions are irreducible since from any position the king can get to any other position on the board. And therefore by theorem 1.19 
$p(x, (1,1))^n \rightarrow \pi((1,1)) = 3/420$. 
Then by theorem 1.21 
\[
E_{(1,1)} T_{(1,1)} = 420/3 = 140
\]


\section*{Q5}
First since the problem was not clear on edge cases I assume $p(1,2) = 1$ and $p(6,5) = 1$ . With that we first realize clearly that all states communicate with state 1 since each contains a shortest path to state 1 of positive probability. 
So we again may use theorem 1.28 where our exit state is 1. 
and solve for $g$
\[
g[2 .. 6] = (I - r)^{-1} \vec{1}
\]
And it turns out $g$ is equal to 
\[
g = [0 , 2.875,   5.625,   8.125,  10.125,  11.125]^T
\]
So since state 1 transitions to 2 with probability 1 then
\[
E_1T_1 = 1 + g(2) = 3.875
\]

\section*{Q6}
Let $C = {G,A}$. Then since $p(G,F) > 0$ and $p(A,B) > 0$ then clearly $\rho_{xF} + \rho_{xB} > 0 $ for both $x = G$ and $x= A$.
Then let $h(x) = P_x(V_F < V_B)$, and using theorem 1.27,
$h(F) = 1, h(B) = 0$, and for $x \in C$,
\[
h(x) = r(x,F) + \sum_y r(x,y)h(y)
\]
where $r$ is $p$ restricted to $\{G,A\}$
Then,

\[
h[G,A] = (I - r)^{-1}v
\]
where, $v$ is a column vector of $p(x,F)$.

$h$ turns out to be
\[
[ 0.875,  0.75 ]^T
\]
Therefore the probability of paying of your debt in full given you are in good standing is 0.875 and 0.75 if you are in arrears. 

\section*{Q7}
With the same logic as Q2 we may assume without loss of generality that our start position is at time 1. Then there are two different requirements to reach all states before returning to one depending on the firs transition. If from the start position we transition to state 2, then we will reach all states before returning to 1 if and only we reach state 12 before reaching state 1. Vice versa if first we transition to state 12 then we will reach every state before returning to the start state if and only if you reach state 2 before returning to state 1. 
Therefore the probabality of reaching all other states before returning to the start state is
\[
p(1,2)*P_2(V_12 < V_1) + p(1,12)*P_12(V_2 < V_1)
\]
Since the probability of transitioning left and right are equal it then 
\[
P_2(V_{12} < V_1) = P_12(V_2 < V_1)
\]
So the probability above becomes simply
\[
P_2(V_{12} < V_1)
\]
Since the states are finite and all communicate with eachother then we may apply theorem 1.27 with $h(1) = 0$ , $h(12) = 1$ and $h(x) = r(x,12) + \sum_y r(x,y)h(y)$ where $r$ is $p$ restricted to $\{2..11\}$. 
We can then solve for $h$

\[
h[2 .. 11] = (I -r)^{-1} v
\]
where $v$ is the column vector with value $r(x,a)$.
$h$ turns to be 
\[
[0, 0.09090909,  0.18181818,  0.27272727,  0.36363636,  0.45454545,
        0.54545455,  0.63636364,  0.72727273,  0.81818182,  0.90909091,1]
\]

So
\[
P_2(V_{12} < V_1) = h(2) = 0.09090909
\]

\section*{Q8}
For this problem there are these finite set of states
\[
\{S, AB, AC, BA, BC, CA, CB, AW, BW, CW\}
\]
where $S$ is the start state where $A$ plays $B$, $xy$ is the state where $x$ just won and is now playing $y$ and $xW$ is the state where $x$ wins. 
The transition matrix then becomes,

\begin{tabular}{|l|l|l|l|l|l|l|l|l|l|l|}
\hline
   & S & AB  & AC  & BA  & BC  & CA  & CB  & AW  & BW  & CW  \\ 
   \hline
S  & 0 & 0   & 0.5 & 0   & 0.5 & 0   & 0   & 0   & 0   & 0   \\ 
\hline
AB & 0 & 0   & 0   & 0   & 0.5 & 0   & 0   & 0.5 & 0   & 0   \\ 
\hline
AC & 0 & 0   & 0   & 0   & 0   & 0   & 0.5 & 0.5 & 0   & 0   \\ 
\hline
BA & 0 & 0   & 0.5 & 0   & 0   & 0   & 0   & 0   & 0.5 & 0   \\ 
\hline
BC & 0 & 0   & 0   & 0   & 0   & 0.5 & 0   & 0   & 0.5 & 0   \\ 
\hline
CA & 0 & 0.5 & 0   & 0   & 0   & 0   & 0   & 0   & 0   & 0.5 \\ 
\hline
CB & 0 & 0   & 0   & 0.5 & 0   & 0   & 0   & 0   & 0   & 0.5 \\ 
\hline
AW & 0 & 0   & 0   & 0   & 0   & 0   & 0   & 1   & 0   & 0   \\ 
\hline
BW & 0 & 0   & 0   & 0   & 0   & 0   & 0   & 0   & 1   & 0   \\ 
\hline
CW & 0 & 0   & 0   & 0   & 0   & 0   & 0   & 0   & 0   & 1 \\ \hline
\end{tabular}

Then Let $X = \{AW, BW\}$, $Y = \{CW\}$ and $Z$ be all other states. 
Then if $h(x) = P(V_Y < V_X)$ , then $h(AW) = h(BW) = 0$, $h(CW) = 1$ and for everything else,
\[
h(x) = r(x,CW) + \sum_y r(x,y)h(y)
\]
Since every state in $Z$ except $S$ can transition into one of the win states and $S$ transitions into $Z \\ S$ with probability 1 then clearly all states in $Z$ communicate with one of the win states so we can apply theorem 1.27 and solve for $h$.
$h$ turns out to be,
\[
[ 0.28571429  0.14285714  0.28571429  0.14285714  0.28571429  0.57142857
  0.57142857, 0, 0, 1]
  \]
  
So the probability of $C$ winning is $h(S) = 0.28571429$

\section*{Q9}
Since we start at state 1,
\[
E_1T_4 = E_1V_4
\]

Then since the states are clearly irreducible since they are all connected then for any state $x$ $P(V_4 < \infty) > 0 $ and we can apply theorem 1.28 to solve for
$
g(x) = E_xV_4
$. 
Cleary $g(4) = 0$  and for all other $x$.
\[
g(x) = 1 + \sum_y p(x,y)g(y)
\]

solving for $g$ we get, 
\[
g = [ 7.18181818,  6.72727273,  5.63636364,  4.36363636,0]
\]

therefore
\[
E_1V_4 = E_1T_4 = 7.1818
\]







\end{document}