\documentclass{article}
\usepackage{amsmath}
\usepackage{amsfonts}
\usepackage{parskip}

\setlength{\parskip}{5pt}
\setlength{\parindent}{0pt}

\title{HW 13}
\date{4/24/17}
\author{Max Horowitz-Gelb}

\begin{document}
\maketitle

\section*{Q1}
First note that for $i \neq j$
$$
q(i,j) = \lim_{t \to 0} \frac{p_t(i,j) u(i,j)}{t}
$$


Then by definition of $p'$ and the Chapman-Kolmogorov equation,
$$
p'_t(i,j) =  \lim_{h \to 0} \frac{p_{t+h}(i,j) - p_t(i,j)}{h}
$$
$$
= \lim_{h \to 0} \frac{\sum_{k} \big(p_t(i,k) p_h(k,j)\big) - p_t(i,j)}{h}
$$
$$
= \lim_{h \to 0} \frac{\sum_{k\neq j} \big(p_t(i,k) p_h(k,j)\big) + p_t(i,j)(p_h(j,j)-1)}{h}
$$
$$
= \lim_{h \to 0} \frac{\sum_{k\neq j} \big(p_t(i,k) p_h(k,j)\big)}{h} + \lim_{h \to 0} \frac{p_t(i,j)(p_h(j,j)-1)}{h}
$$

$$
= \sum_{k\neq j} p_t(i,k) q(k,j) - \lim_{h \to 0} \frac{p_t(i,j)(\sum_{k\neq j} p_h(j,k)}{h}
\qquad \text{By definition of }q
$$
$$
= \sum_{k\neq j} p_t(i,k) q(k,j)  + p_t(i,j)(-\sum_{k\neq j} q(j,k))
$$
$$
= \sum_{k\neq j} p_t(i,k) q(k,j)  + p_t(i,j)(-\lambda_j)
$$
$$
= p_t(i,\cdot) Q(\cdot,j)
$$

\section*{Q2}
Note
$$
\frac{\delta}{\delta t} p_t = \frac{\delta}{\delta t} e^{Qt}
$$
$$
= Q \sum_{n=1} ^\infty \frac{n Q^{n-1} t^{n-1}}{n!}
$$
$$
= Q \sum_{n=1} ^\infty \frac{Q^{n-1} t^{n-1}}{(n-1)!} 
$$
$$
= Q p_t \qquad \text{by the givien solution of KBE}
$$

\section*{Q3}

$$
(\tilde{\pi}\tilde{p})_i = \frac{1}{\sum_{k\in S} \pi_k\lambda_k} \sum_{k \neq i} \pi_k \lambda_k \tilde{p}(k,i)
$$
$$
= \frac{1}{\sum_{k\in S} \pi_k\lambda_k} \sum_{k \neq i} \pi_k \lambda_k \frac{q(k,i)}{\lambda_k}
$$
$$
= \frac{1}{\sum_{k\in S} \pi_k\lambda_k} \sum_{k \neq i} \pi_k q(k,i)
$$
We know that $\pi Q = 0$ so This is equal to 
$$
\frac{-\pi_i Q_{i,i}}{\sum_{k\in S} \pi_k\lambda_k} 
$$
$$
= \frac{\pi_i \lambda_i}{\sum_{k\in S} \pi_k\lambda_k} 
$$
so $\tilde{\pi}$ is a stationary measure, and clearly by the normalizing term it is also a distribution.

\section*{Q4}

\end{document}