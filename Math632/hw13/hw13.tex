\documentclass{article}
\usepackage{amsmath}
\usepackage{amsfonts}
\usepackage{parskip}

\setlength{\parskip}{5pt}
\setlength{\parindent}{0pt}

\title{HW 13}
\date{4/24/17}
\author{Max Horowitz-Gelb}

\begin{document}
\maketitle

\section*{Q1}
First note that for $i \neq j$
$$
q(i,j) = \lim_{t \to 0} \frac{p_t(i,j) u(i,j)}{t}
$$


Then by definition of $p'$ and the Chapman-Kolmogorov equation,
$$
p'_t(i,j) =  \lim_{h \to 0} \frac{p_{t+h}(i,j) - p_t(i,j)}{h}
$$
$$
= \lim_{h \to 0} \frac{\sum_{k} \big(p_t(i,k) p_h(k,j)\big) - p_t(i,j)}{h}
$$
$$
= \lim_{h \to 0} \frac{\sum_{k\neq j} \big(p_t(i,k) p_h(k,j)\big) + p_t(i,j)(p_h(j,j)-1)}{h}
$$
$$
= \lim_{h \to 0} \frac{\sum_{k\neq j} \big(p_t(i,k) p_h(k,j)\big)}{h} + \lim_{h \to 0} \frac{p_t(i,j)(p_h(j,j)-1)}{h}
$$

$$
= \sum_{k\neq j} p_t(i,k) q(k,j) - \lim_{h \to 0} \frac{p_t(i,j)(\sum_{k\neq j} p_h(j,k)}{h}
\qquad \text{By definition of }q
$$
$$
= \sum_{k\neq j} p_t(i,k) q(k,j)  + p_t(i,j)(-\sum_{k\neq j} q(j,k))
$$
$$
= \sum_{k\neq j} p_t(i,k) q(k,j)  + p_t(i,j)(-\lambda_j)
$$
$$
= p_t(i,\cdot) Q(\cdot,j)
$$

\section*{Q2}
Note
$$
\frac{\delta}{\delta t} p_t = \frac{\delta}{\delta t} e^{Qt}
$$
$$
= Q \sum_{n=1} ^\infty \frac{n Q^{n-1} t^{n-1}}{n!}
$$
$$
= Q \sum_{n=1} ^\infty \frac{Q^{n-1} t^{n-1}}{(n-1)!} 
$$
$$
= Q p_t \qquad \text{by the givien solution of KBE}
$$

\section*{Q3}

$$
(\tilde{\pi}\tilde{p})_i = \frac{1}{\sum_{k\in S} \pi_k\lambda_k} \sum_{k \neq i} \pi_k \lambda_k \tilde{p}(k,i)
$$
$$
= \frac{1}{\sum_{k\in S} \pi_k\lambda_k} \sum_{k \neq i} \pi_k \lambda_k \frac{q(k,i)}{\lambda_k}
$$
$$
= \frac{1}{\sum_{k\in S} \pi_k\lambda_k} \sum_{k \neq i} \pi_k q(k,i)
$$
We know that $\pi Q = 0$ so This is equal to 
$$
\frac{-\pi_i Q_{i,i}}{\sum_{k\in S} \pi_k\lambda_k} 
$$
$$
= \frac{\pi_i \lambda_i}{\sum_{k\in S} \pi_k\lambda_k} 
$$
so $\tilde{\pi}$ is a stationary measure, and clearly by the normalizing term it is also a distribution.

\section*{Q4}
\subsection*{i}
Note
$$
\lim_{t \to \infty} P_t(B,C) = \pi_C
$$

Then we solve for
$$
\pi \begin{bmatrix}
Q_{A,A} & . & . \\
Q_{B,A} & . & . \\
1 & 1 & 1
\end{bmatrix}
= [0, 0, 1]
$$


and we get $\pi = [ 1/2,1/4, 1/4]$
so $\pi_C = 1/4$.

\subsection*{ii}
The limiting fraction for the time in each city is simply the stationary distribution.
So
the limiting fraction for Atlanta is $1/2$, the limiting fraction for Boston is $1/4$ and the limiting fraction for Chicago is $1/4$.

\subsection*{iii}
Let $N(t)$ be the number of cities visited $t$ years, $X_{i}$ be the amount of time spent in the $i^{th}$ city and $Z_{B_i} = 1$ if city $i$ is Boston.

Then by SLLN,
$$
\lim_{t \to \infty} N(t)^{-1} \sum_{i =1}^{N(t)} Z_{B,i}X_{i} 
$$
$$
=\lim_{t \to \infty} N(t)^{-1} \sum_{i=1}^{N(t)} Z_{B,i}\sum_{i =1}^{N_B(t)} X_{B,i} = \lim_{n \to \infty} n^{-1} \sum_{i =1}^{n} X_{B,i} = E[X_B] = 1/4
$$
So
$$
E[\sum_{i=1}^{N(1)} Z_{B,i}\sum_{i =1}^{N_B(1)} X_{B,i}] = 1/4
$$
$$
E[\sum_{i=1}^{N(1)} Z_{B,i}] \pi_B = 1/4
$$
$$
E[\sum_{i=1}^{N(1)} Z_{B,i}] 1/4 = 1/4
$$
$$
E[\sum_{i=1}^{N(1)} Z_{B,i}] = 1
$$

\subsection*{iv}
The expected number of times she flies from Boston to Atlanta is
$$
E[\sum_{i=1}^{N(1)} Z_{B,i}] * P(B,A) = 3/4
$$

\section*{Q5}

$$
Q = \begin{tabular}{c | c | c |c | c|}

& 0 & 1 & 2 & 3 \\
\hline
0 & -1 & 0 & 1 & 0\\
\hline
1 & 2 & -3 & 0 & 1 \\
\hline
2 & 0 & 2 & -2 & 0 \\
\hline
3 & 0 & 0 & 2 & -2\\
\hline
\end{tabular}
$$

Solving for $\pi Q = 0$ we get

$$
\pi = [ 0.4,  0.2,  0.3,  0.1]
$$

\section*{Q6}
\subsection*{a}
There are 5 states in this model,

$0$ is when no machines are broken.

$1$ is when machine 1 is broken.

$2$ is when machine 2 is broken.

$12$ is when both machines are broken but $1$ broke first.

$21$ is when both machines are broken but $2$ broke first.

$0$ can transition to 1 or 2

1 can transition to 0, or 12

12 can transition to 2

21 can transition to 1

\subsection*{b}
This gives a $Q$ matrix of
$$
\begin{tabular}{llllll}
                        & 0                       & 1                       & 2                       & 12                      & 21                      \\ \cline{2-6} 
\multicolumn{1}{l|}{0}  & \multicolumn{1}{l|}{-4} & \multicolumn{1}{l|}{1}  & \multicolumn{1}{l|}{3}  & \multicolumn{1}{l|}{0}  & \multicolumn{1}{l|}{0}  \\ \cline{2-6} 
\multicolumn{1}{l|}{1}  & \multicolumn{1}{l|}{2}  & \multicolumn{1}{l|}{-5} & \multicolumn{1}{l|}{0}  & \multicolumn{1}{l|}{3}  & \multicolumn{1}{l|}{0}  \\ \cline{2-6} 
\multicolumn{1}{l|}{2}  & \multicolumn{1}{l|}{4}  & \multicolumn{1}{l|}{0}  & \multicolumn{1}{l|}{-5} & \multicolumn{1}{l|}{0}  & \multicolumn{1}{l|}{1}  \\ \cline{2-6} 
\multicolumn{1}{l|}{12} & \multicolumn{1}{l|}{0}  & \multicolumn{1}{l|}{0}  & \multicolumn{1}{l|}{2}  & \multicolumn{1}{l|}{-2} & \multicolumn{1}{l|}{0}  \\ \cline{2-6} 
\multicolumn{1}{l|}{21} & \multicolumn{1}{l|}{0}  & \multicolumn{1}{l|}{4}  & \multicolumn{1}{l|}{0}  & \multicolumn{1}{l|}{0}  & \multicolumn{1}{l|}{-4} \\ \cline{2-6} 
\end{tabular}
$$

Solving for $\pi Q = 0$
we get
$$
\pi = [ 0.34108527,  0.12403101,  0.27906977,  0.18604651,  0.06976744]
$$
\end{document}