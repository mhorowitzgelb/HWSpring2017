\documentclass{article}
\usepackage{amsmath}
\usepackage{amsfonts}

\title{HW 6}
\date{2/27/17}
\author{Max Horowitz-Gelb}

\begin{document}
\maketitle
\section*{Q1}
Let our state space be $[T,C]$. Then we can think of one vehicle following another vehicle as a transition 
with transition matrix,
\[
p_{i,j} =
\begin{bmatrix}
1/4 & 3/4\\
1/5 & 4/5
\end{bmatrix}
\]
We can then solve for the stationary distribution of this transition matrix simply since it is a 2x2 matrix.

\[
\pi_T = \frac{1/5}{1/5 + 3/4} = 0.2105, 
\pi_C = \frac{3/4}{1/5 + 3/4} = 0.7895
\]

Then since $p_{i,j} \neq 0$ for all $i,j$, then clearly $p$ is irreducible and aperiodic. Then since our state space is finite and we have a stationary distribution $\pi$, then theorem 1.19 says the probability of being a truck converges to $\pi_T = 0.2105$. 

\section*{Q2}
For this we simply solve for the set of functions
\[
\pi_1 - (0.86\pi_1 + 0.05\pi_2 + 0.03\pi_3) = 0
\]
\[
	\pi_2 - (0.08\pi_1 + 0.88\pi_2 + 0.05 \pi_3)= 0
\]
\[
	\pi_1 + \pi_2 + \pi_3 = 1
\]

Then,
\[
\pi = [0,0,1]\begin{bmatrix}
0.14 & -0.08 & 1 \\
-0.05 & 0.12 & 1\\
-0.04 & -0.05 & 1
\end{bmatrix}^{-1} 
 = [ 0.2155477 ,  0.33215548,  0.45229682]
\]

Then since $p$ is finite and $p_{i,j} \neq 1$ for all $i,j$ then $p$  is irreducible and is aperiodic, we may again apply theorem 1.19 to say that in the long run,
$\pi_1 = 0.2155477$ fraction of the population will reside in cities, $\pi_2 = 0.33215548$ fraction of people will reside in suburbs and $\pi_3 = 0.45229682$ fraction of people will reside in rural areas.

\section*{Q3}
\subsection*{a.}
Using the same strategy as Q2 we first solve for a stationary distribution to get
\[
\pi = [ 0.36363636,  0.35664336,  0.27972028]
\]
 
Then since p is finite and $p_{i,j} \neq 0$ for all $i,j$ then we again can apply 1.19 to say,  
\[
\lim_{n \to \infty} p^n(1,2) = 0.35664336
\]
\subsection*{b.}
Then since $p$ is irreducible and closed, then it is recurrent, and we may apply theorem 1.21 which says,
\[
\frac{N_n(2)}{n} \rightarrow \frac{1}{E_2T_2}
\]
and as shown in a.,
\[
\frac{N_n(2)}{n} \rightarrow \pi_2
\]
so ,
\[
\pi_2 = \frac{1}{E_2T_2}
\]
and
\[
E_2T_2 -1 = 1/\pi_2 -1 = 1.8039
\]
\subsection*{c.}
\[
E[\text{walking distance}] = \sum_{n} p(n) * (0.3 + 0.1n)
\]
\[
= \sum_{n} \pi_n * (0.3 + 0.1n) = 0.4916
\]

\section*{Q4}
\subsection*{i}
Our state space is $[1,2,2']$ where the number indicates the number of light bulbs working and ' indicates we just replaced  the light bulbs.
Then our transition matrix is
\[
\begin{bmatrix}
0.95 & 0 & 0.05\\
0.02 & 0.98 & 0\\
0.02 & 0.98 & 0 
\end{bmatrix}
\]
Solving for the stationary distribution we get,
\[
\pi = [ 0.2857,  0.7       ,  0.0143]
\]
Then since $p$ is finite and nonzero for all $i,j$, then it is irreducible and theorem 1.19 says that the long run fraction of time one light bulb is working is $\pi_1 = 0.2857$. 

\subsection*{ii}
Then we can use theorem 1.21 to say that
\[
E_2T_{2'} - 1 = 1/\pi_{2'} -1 = 69
\]
So the expected number of days between light bulb is 69 days.

\section*{Q5}
First since states $\{1 ... M \}$ are closed, then for any $x \in \{1 ... M\}$. 
\[
\sum_m p(x,m) = \sum_m f((x-m) \mod M) = \sum_{n = 0}^{M-1} f(n) = 1
\]
Then for any $x \in \{1 ... M\}$,
\[
\sum_m p(m,x) = \sum_m f((m-x) \mod M) = \sum_{n = 0}^{M-1} f(n) = 1
\]
therefore the THMC is doubly stochastic implying at has a uniform stationary distribution
\[
\pi = [1/M ... 1/M]
\]
Clearly our state space is finite. Then since $p(1,2) > 0$ and $p(2,4) > 0)$
then for all $x$, $p(x,(x+1) \mod M) , p(x, (x-1) \mod M), p(x, (x+2) \mod M) ,  p(x, (x-2) \mod M)$ are all positive. So for any integers $a,b$ such that $a + 2b = M$, $p^{a+b}(x,x) \neq 0$. For any $M \geq 4$ the set of $\{a+b\}$ has a gcd of 1. Therefore our THMC is aperiodic and by theorem 1.19
\[
lim_{n \to \infty} p^n(1,1) = \pi_1 = 1/M
\]

Then using theorem 1.23
\[
lim_{n \to \infty} \sum_{m = 1}^n X_m^2 = \sum_{i= 1} ^M \pi_i * i^2 = 1/M * \sum_{i = 1}^M i^2
\]

\section*{Q6}
2 already shows us that $v$ is a distribution so we must only show that $v$ is a stationary measure. 
Note
\[
(v p)_j = \sum_i v_ip(i,j)   
\]
which then using 3.
\[
 = \sum_i v_jp(j,i) = v_j * \sum_i p(j,i) = v_j
\]

Therefore $v$ is a stationary distribution.

\section*{Q7}
\subsection*{a.}
First we solve for a stationary distribution
$
\pi = [ 0.5854,  0.2927,  0.0975,  0.0244]
$
Clearly $\{X_n\}$ is closed and since $p(1,1) > 0$ and for all $x$, $x \rightarrow 1$ then all states are aperiodic and by theorem 1.19 
\[
\lim_{n \to \infty} p(x,1) = \pi_1 = 0.5854
\]
which is the fraction of the time Dave shaves in the long run.

\subsection*{b.}
Yes $\pi$ has detailed balance condition after verifying,
for all $i,j$
\[
\pi_i*p_{i,j} = \pi_j*p{j,i}
\]

\section*{Q8}
\subsection*{a.}
\[p(i,j) = 
\begin{cases}
\frac{i*(b-i)}{m^2} + \frac{(m-i)(m-b+i)}{m^2} & i = j\\
\frac{(m-i)(b-i)}{m^2} & i = j - 1\\
\frac{i(m-b+i)}{m^2} & i= j +1 \\
0 & else\\
\end{cases}
\] 

\subsection*{b.}
$\pi$ is a stationary distribution since it satisfies the detailed balance condition. 
This can be shown with 3 cases 
\subsubsection*{Case 1 $|i -j| \geq 2$}

Then clearly
\[
\pi_i*p(i,j) = \pi_j * p(j,i) = 0
\]

\subsubsection*{Case 2 $i = j$}
Then clearly
\[
\pi_i*p(i,j) = \pi_j * p(j,i)
\]
\subsubsection*{Case 3 $|i-j| = 1$}
Without loss of generality assume that $i = j -1$ since we can simply flip our equation around. 

Then, 
\[
\pi_i*p(i,j) = \binom{b}{i}\binom{2m-b}{m-i} \big{/} \binom{2m}{m} \frac{(m-i)(b-i)}{m^2}
\]
\[
= \frac{b!}{i!(b-i-1)!}\frac{(2m-b)!}{(m-i-1)!(m-b + i)!} \frac{\binom{2m}{m}}{m^2}
\]
and,
\[
\pi_j*p(j,i) = \binom{b}{j}\binom{2m-b}{m-j} \big{/} \binom{2m}{m} \frac{j(m-b+j)}{m^2}
\]
\[
\frac{b!}{(j-1)!(b-j)!}\frac{(2m-b)!}{(m-j)!(m-b + j - 1)!} \frac{\binom{2m}{m}}{m^2}
\]
Then since $j = i +1$, plugging in $i+1$ you see that both equations are identical or,
\[
\pi_i*p(i,j) = \pi_j * p(j,i)
\]

Therefore $\pi$ satisfies the detailed balance condition and is a stationary distribution.
\end{document}