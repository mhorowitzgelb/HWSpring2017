\documentclass{article}

\usepackage{amsmath}
\usepackage{amsfonts}

\usepackage{parskip}

\setlength{\parskip}{5pt}
\setlength{\parindent}{0pt}

\title{HW 10}
\author{Max Horowitz-Gelb}
\date{4/2/2017}

\begin{document}
\maketitle

\section*{Q1}
\subsection*{a}
Let $X_h$ and $X_m$ distributed as exponentials with rate $\lambda_h, \lambda_m$, be the finish times  of the haircut and manicure respectively. 
Since $X_h \bot X_m$, then,
$$
P(X_m < X_h) = \frac{\lambda_m}{\lambda_m + \lambda_h} = \frac{1/20}{1/20 + 1/30} = 0.6
$$

\subsection*{b}
The expected amount of time before they both finish is 
$$
E[max(X_m, X_h)] = P(X_m < X_h) * E[X_h | X_m < X_h] + P(X_h < X_m) * E[X_m | X_h < X_m] 
$$
which because of independence is equal to ,
$$
P(X_m < X_h) * E[X_h] + P(X_h < X_m) * E[X_m] = 0.6 * 30 + 0.4 *20 =  26
$$

\section*{Q1}
\subsection*{a}
Since the exponential distribution is memory-less and the handyman acts independently of when the janitor last replaces the bulb then when a light-bulb is replaced, regardless of who replaces it, it's a Poisson race between the janitor and the handyman for who replaces the next bulb first. Therefore we can sum they're respective rates to get the total rate of bulbs replaced. 
This is,
$$
\lambda = \lambda_h + \lambda_j = 1/200 + 0.01 = 0.015
$$

\subsection*{b}
Since it is a Poisson race be the janitor and the handyman replacing the light, then the fraction of times in the long run the the janitor replaces a bulb that just broke is,
$$
\frac{\lambda_j}{\lambda_j + \lambda_h} = \frac{1/200}{1/200 + 0.01} = 1/3
$$



\end{document}