\documentclass{article}
\usepackage{amsmath}
\usepackage[a4paper, total={7in, 8in}]{geometry}
\title{Homework 2}
\author{Max Horowitz-Gelb}
\begin{document}
\maketitle
\section*{Q1}
\subsection*{a.}
Are transition matrix $p_{i,j} = $
\begin{tabular}{|l|l|l|l|}
\hline
- & A & B & C\\ \hline
A & 0 & $1/2$ & $1/2$\\ \hline
B & 3/4 & 0 & 1/4\\ \hline
C & 3/4 & 1/4 & 0\\ \hline
\end{tabular}

\subsection*{b.}
To calculate probabilities of events at time 2 we simply calculate $p(i,j)^2$.

$p(i,j)^2 = $ 
\begin{tabular}{|l|l|l|l|}
\hline
- & A & B & C\\ \hline
A & 3/4 & $1/8$ & $1/8$\\ \hline
B & 3/16 & 7/16 & 3/8\\ \hline
C & 3/16 & 3/8 & 7/16\\ \hline
\end{tabular}

So,
\[
P(X_2 = A | X_0 = A) = p(A,A)^2 = 3/4
\]
\[
P(X_2 = B | X_0 = A) = p(A,B)^2 = 1/8
\]
\[
P(X_2 = C | X_0 = A) = p(A,C)^2 = 1/8
\]
Finally we can also calculate
\[
P(X_3 = B | X_0 = A) = p(A,A)^2 * p(A,B) + p(A,B)^2 * p(B,B) + p(A,C)^2 * p(C,B)
\]
\[
= 3/4 * 1/2 +  1/8 * 0 + 1/8 * 1/4 =  13/32
\]

\section*{Q2}
\[
P(X_2 = 3, X_4 = 4 | X_7 = 9, X_6 = 8)
\]
\[
= \frac{P(X_2 = 3, X_4 =4 ,  X_7 = 9, X_6 = 8)}{P( X_7 = 9, X_6 = 8)} \text{, using basic defintition of conditional probability}
\]
\[
 = \frac{P(X_7 = 9 | X_6 = 8) * P(X_6=8|X_4 = 4) * P(X_4 =4 | X_2 = 3) * P(X_2=3)}{P(X_7= 9 | X_6 = 8) * P(X_6 = 8)} \text{, using that} X \text{ is  a THMC and chain rule}
\]
\[
 = \frac{p(8,9)*p(4,8)^2*p(3,4)^2*p(1,3)^2}{p(8,9)*p(1,8)^6}
 \]
\end{document}

